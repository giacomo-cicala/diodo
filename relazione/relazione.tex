\documentclass[12pt,italian]{article}
\usepackage[italian]{babel}
\usepackage[a4paper, margin=1.97cm]{geometry}
\usepackage{graphicx}
\usepackage{amsmath}
\usepackage{circuitikz}
\usepackage{caption}
\usepackage{amsfonts}

\usepackage[hidelinks]{hyperref}
\usepackage{cleveref}

\newcommand{\err}[1]{\textcolor{red}{#1}}

\title{Misura della caratteristiva I-V di due diodi a giunzione p-n}
\author{Enrico Barbuio \\ 0001117553 \and Giacomo Cicala \\ 0001122965} 
\date{\today}

\begin{document}
\maketitle

\renewcommand{\abstractname}{Abstract}
\begin{abstract}
    L'esperimento ha avuto come obbiettivo la realizzazione di un circuito per la misura delle curve caratteristica I-V di due diodi a giunzione p-n, uno al silicio e uno al germanio. I valori della corrente di saturazione inversa e del prodotto del fattore di idealità con la tensione termica per il diodo al silicio sono risultati essere
    \begin{equation*}
        (I_{0})_{Si} = ( \cdots \pm \cdots) \text{ mA} \hspace{2cm} (\eta V_{T})_{Si} = (\cdots \pm \cdots) \text{ mV}
    \end{equation*}

    \noindent
    mentre per il diodo al germanio sono risultati essere

    \begin{equation*}
        (I_{0})_{Ge} = ( \cdots \pm \cdots) \text{ mA} \hspace{2cm} (\eta V_{T})_{Ge} = (\cdots \pm \cdots) \text{ mV}
    \end{equation*}

\end{abstract}

\section*{Introduzione}

Nell' esperimento svolto in laboratorio abbiamo abbiamo realizzato un circuito (\cref{fig:schema_elettrico}) per la misura della caratteristica I-V di due diodi a giunzione p-n a polarizzazione diretta, uno al silicio e uno al germanio. La caratteristica I-V di un diodo non ideale è descritta dall'equazione di Shockley:

\begin{equation}
    I = I_{0} \left( e^{\frac{V}{\eta V_{T}}} - 1 \right)
    \label{eq:shockley}
\end{equation}

\noindent
dove $I$ è la corrente che attraversa il diodo, $V$ è la tensione ai suoi capi, $I_{0}$ è la corrente di saturazione inversa e $V_{T}$ è la tensione termica. Il termine $\eta$ è il fattore di idealità: questo tiene conto delle deviazioni del diodo reale rispetto al modello teorico ideale di Shockley, dovute principalmente a fenomeni di generazione termica e ricombinazione all'interno della depletion region.

\begin{figure}[h!]
    \centering
    \begin{circuitikz}[scale=1]
        \% 1. Generatore di Tensione e collegamento al Potenziometro
    % Disegno dal basso (0,0) verso l'alto (0,3) e poi verso destra
    \draw (0,0) to[V, v=$V_{in}$] (0,3) 
                -- (2,3);

    % 2. Il Potenziometro (Verticale)
    % TRUCCO: Disegnandolo dall'alto (2,3) verso il basso (2,0),
    % il "wiper" (cursore) si troverà automaticamente sul lato destro.
    \draw (2,3) to[potentiometer, n=P, l_=$R$] (2,0)
                -- (0,0); % Chiudo il loop verso il generatore
    
    % Aggiungo la massa
    \draw (0,0) node[ground]{};

    % 3. Il ramo del Diodo
    % Parto dall'ancoraggio del cursore (P.wiper)
    \draw (P.wiper) -- ++(1.5,0)        % Filo orizzontale verso destra
                    to[D, l=$D$] ++(0,-1.5) % Diodo verso il basso (coordinate relative)
                    -- (2,0);           % Chiudo il circuito sulla linea di massa
        

    \end{circuitikz}
    \caption{Schema elettrico del circuito utilizzato per la misura della caratteristica I-V di un diodo a giunzione p-n.}
    \label{fig:schema_elettrico}
\end{figure}

\section*{Apparato sperimentale e svolgimento}

\section*{Risultati e discussione}

\section*{Conclusioni}

\appendix
\section{Appendici}

\end{document}