\documentclass[12pt,italian]{article}
\usepackage[italian]{babel}
\usepackage[a4paper, margin=1.97cm]{geometry}
\usepackage{graphicx}
\usepackage{amsmath}
\usepackage{circuitikz}
\usepackage{caption}
\usepackage{subcaption}
\usepackage{amsfonts}
\usepackage[hidelinks]{hyperref}
\usepackage{cleveref}
\usepackage{siunitx}
\usepackage{booktabs}
\graphicspath{{./images/}}

\newcommand{\err}[1]{\textcolor{red}{#1}}
\crefname{table}{tab.}{tab.}

\title{Misura della caratteristiva I-V di diodi a giunzione p-n al germanio e al silicio}
\author{Enrico Barbuio \\ 0001117553 \and Giacomo Cicala \\ 0001122965} 
\date{\today}

\begin{document}
\maketitle

\renewcommand{\abstractname}{Abstract}
\begin{abstract}
    L'esperimento ha avuto come obbiettivo la realizzazione di un circuito per la misura delle curve caratteristica I-V di due diodi a giunzione p-n, uno al silicio e uno al germanio. I valori della corrente di saturazione inversa e del prodotto del fattore di idealità con la tensione termica per il diodo al silicio sono risultati essere
    \begin{equation*}
        (I_{0})_{Si} = ( \cdots \pm \cdots) \text{ mA} \hspace{2cm} (\eta V_{T})_{Si} = (\cdots \pm \cdots) \text{ mV}
    \end{equation*}

    \noindent
    mentre per il diodo al germanio sono risultati essere

    \begin{equation*}
        (I_{0})_{Ge} = ( \cdots \pm \cdots) \text{ mA} \hspace{2cm} (\eta V_{T})_{Ge} = (\cdots \pm \cdots) \text{ mV}
    \end{equation*}

\end{abstract}

\section*{Introduzione}

Nell' esperimento svolto in laboratorio abbiamo abbiamo realizzato un circuito (\cref{fig:schema_elettrico}) per la misura della caratteristica I-V di due diodi a giunzione p-n a polarizzazione diretta, uno al silicio e uno al germanio. La caratteristica I-V di un diodo non ideale è descritta dall'equazione di Shockley:

\begin{equation}
    I = I_{0} \left( e^{\frac{V}{\eta V_{T}}} - 1 \right)
    \label{eq:shockley}
\end{equation}

\noindent
dove $I$ è la corrente che attraversa il diodo, $V$ è la tensione ai suoi capi, $I_{0}$ è la corrente di saturazione inversa e $V_{T}$ è la tensione termica. Il termine $\eta$ è il fattore di idealità: questo tiene conto delle deviazioni del diodo reale rispetto al modello teorico ideale di Shockley, dovute principalmente a fenomeni di generazione termica e ricombinazione all'interno della depletion region.

\begin{figure}[h!]
    \centering
    \begin{circuitikz}[scale=1]
        % 1. Generatore di Tensione e collegamento al Potenziometro
        % Disegno dal basso (0,0) verso l'alto (0,3) e poi verso destra
        \draw (0,0) to[american voltage source,invert, l=$+5V
            $] (0,3)
        -- (2,3);

        % 2. Il Potenziometro (Verticale)
        % TRUCCO: Disegnandolo dall'alto (2,3) verso il basso (2,0),
        % il "wiper" (cursore) si troverà automaticamente sul lato destro.
        \draw (2,3) to[potentiometer, n=P, l_=$R$] (2,0)
        -- (0,0); % Chiudo il loop verso il generatore

        % Aggiungo la massa
        \draw (0,0) node[ground]{};

        % 3. Il ramo del Diodo
        % Parto dall'ancoraggio del cursore (P.wiper)
        \draw (P.wiper) -- ++(1.5,0)        % Filo orizzontale verso destra
        to[D*, l=$D$] ++(0,-1.5) % Diodo verso il basso (coordinate relative)
        -- (2,0);           % Chiudo il circuito sulla linea di massa


    \end{circuitikz}
    \caption{Schema elettrico del circuito utilizzato per la misura della caratteristica I-V di un diodo a giunzione p-n.}
    \label{fig:schema_elettrico}
\end{figure}

\section*{Apparato sperimentale e svolgimento}

Il circuito (\cref{fig:schema_elettrico}) è stato realizzato su una piastra forata (\err{foto?}) utilizzando un alimentatore a bassa tensione \textit{TTi EB2025T} impostato a  $+5$V(\cref{fig:alimentatore}), un potenziometro da 1 k$\Omega$ e due diodi a giunzione p-n, uno al silicio (\err{codice diodo?}) e uno al germanio (\err{uguale}). La tensione ai capi del diodo e la corrente che lo attraversa sono state misurate rispettivamente con un oscilloscopio  da banco \textit{GW Instek GOS-652G} (\cref{fig:oscilloscopio})  e  da un multimetro \textit{Fluke 175 True RMS Multimeter} (\cref{fig:multimetro}) in modalità amperometro. Prima di verificare l'equazione di Shockley, per testare la calibrazione tra multimetro e oscilloscopio, si è rimosso il diodo (cortocircuitando il ramo) e si sono fatte misure di tensione tra i punti A e B (\err{aggiungere}) del circuito (\cref{fig:schema_elettrico}) con entrambi gli strumenti, variando il valore del potenziometro. Successivamente, per ricavare le caratteristiche I-V, la resistenza del potenziometro è stata variata in modo da ottenere diverse coppie di valori di tensione e corrente.


\begin{figure}[h!]
    \centering
    \begin{subfigure}[b]{0.32\textwidth}
        \centering
        \includegraphics[width=\linewidth]{images/alimentatore.png}
        \caption{Alimentatore}
        \label{fig:alimentatore}
    \end{subfigure}
    \hfill
    \begin{subfigure}[b]{0.32\textwidth}
        \centering
        \includegraphics[width=\linewidth]{images/oscilloscopio3.png}
        \caption{Oscilloscopio}
        \label{fig:oscilloscopio}
    \end{subfigure}
    \hfill
    \begin{subfigure}[b]{0.32\textwidth}
        \centering
        \includegraphics[width=\linewidth]{images/multimetro.png}
        \caption{Multimetro}
        \label{fig:multimetro}
    \end{subfigure}
    \caption{Strumentazione usata nell'apparato sperimentale: (a) Alimentatore  \textit{TTi EB2025T}; (b) Oscilloscopio \textit{GW Instek GOS-652G}; (c) Multimetro \textit{Fluke 175 True RMS Multimeter}.}
    \label{fig:apparato_sperimentale}
\end{figure}

\section*{Risultati e discussione}
Si riportano in (\cref{tab:calibrazione_strumenti}) i valori di tensione misurati con oscilloscopio e multimetro per effettuare la calibrazione degli strumenti.
\sisetup{
    table-number-alignment = center,
    table-figures-integer = 4,
    table-figures-decimal = 8
}

\begin{table}[ht]
\centering
\begin{tabular}{
    S[table-format=4.2]
    S[table-format=4.2]
    S[table-format=3.0]
    S[table-format=3.0]
    S[table-format=2.6]
}
\toprule
\multicolumn{1}{c}{\parbox{2.2cm}{Tensioni multimetro (mV)}} &
\multicolumn{1}{c}{\parbox{2.4cm}{Tensioni oscilloscopio (mV)}} &
\multicolumn{1}{c}{\parbox{3cm}{F.S oscilloscopio (mV/Div)}} &
\multicolumn{1}{c}{\parbox{2.6cm}{Risoluzione oscilloscopio (mV)}} &
\multicolumn{1}{c}{\parbox{3.6cm}{Errori oscilloscopio (mV)}} \\
\midrule
56.80  & 52.0  & 20  & 2  & 2.58526594  \\
145.30 & 140.0 & 50  & 5  & 6.54904573  \\
228.80 & 220.0 & 50  & 5  & 8.29517932  \\
298.00 & 300.0 & 100 & 10 & 13.462912   \\
377.30 & 380.0 & 100 & 10 & 15.1726728  \\
453.70 & 460.0 & 100 & 10 & 17.0496334  \\
531.60 & 540.0 & 100 & 10 & 19.0444218  \\
635.2 & 640.0 & 200 & 20 & 27.7288658  \\
709.0 & 720.0 & 200 & 20 & 29.4416372  \\
815.0 & 800.0 & 200 & 20 & 31.2449996  \\ 
\bottomrule
\end{tabular}

\caption{Tabella (\err{numeri a caso, ricontrolla, mettere simboli titoli?, cifre decimali?}).}
\label{tab:calibrazione_strumenti}
\end{table}

Le incertezze del multimetro sono state omesse in quanto trascurabili rispetto a quelle dell'oscilloscopio.
La risoluzione dell'oscilloscopio (e quindi l'errore sulla lettura) $\sigma_{l}$ è stata calcolata come
\begin{equation}
    \sigma_{l} = \frac{F.S}{5}*(\text{\#tacchette apprezzabili})
\end{equation}
dove in questo caso, in tutte le misure dell'esperimento, il numero di tacchette apprezzabili è stato 0.5. Infine, l'errore totale associato ad ogni misura di tensione con l'oscilloscopio è stato calcolato come

\begin{equation}
    \sigma = \sqrt{(\sigma_{l})^{2} + (\sigma_{c})^{2}}
\end{equation}

\noindent
dove l'errore del costruttore è

\begin{equation}
    \frac{\sigma_{c}}{V_{mis}} = 3\%
\end{equation} 

\noindent
con $V_{mis}$ tensione misurata; l'errore sullo zero dell'oscilloscopio è stato trascurato in quanto si è verificato che fosse molto più piccolo rispetto agli altri errori, grazie ad un opportuno fondo scala.

Eseguendo un fit lineare dei dati di calibrazione (\cref{fig:fit_calibrazione})
\begin{equation}
    V_{osc} = a + b \cdot V_{mult}
\end{equation}
si è ottenuta la seguente relazione tra le due misure di tensione

\begin{figure}[h!]
    \centering
    \includegraphics[width=0.7\textwidth]{example-image}
    \caption{Fit lineare dei dati di calibrazione tra multimetro e oscilloscopio.\err{completare}}
    \label{fig:fit_calibrazione}
\end{figure}

\noindent
I parametri del fit 

\begin{equation*}
    a = \cdots \pm \cdots \text{ mV} \hspace{2cm} b = \cdots \pm \cdots
\end{equation*}

\noindent
indicano che il coefficiente angolare è compatibile con l'unità, mentre l'intercetta non è compatibile con lo zero, indicando un offset che sarà poi corretto nelle misure successive di tensione con l'oscilloscopio.



\section*{Conclusioni}

\appendix
\section{Appendici}

\end{document}