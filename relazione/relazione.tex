\documentclass[12pt,italian]{article}
\usepackage[italian]{babel}
\usepackage[a4paper, margin=1.97cm]{geometry}
\usepackage{graphicx}
\graphicspath{{./images/}}
\usepackage{amsmath}
\usepackage[american]{circuitikz}
\usepackage{caption}
\usepackage{subcaption}
\numberwithin{table}{section}
\usepackage{amsfonts}
\usepackage[hidelinks]{hyperref}
\usepackage{cleveref}
\usepackage{siunitx}
\usepackage{booktabs}
\usepackage{pgfplotstable}
\pgfplotsset{compat=1.8}
\pgfplotstableset{
	search path={../macro/data},
	physicsTable/.style={
        col sep=space,       
        header=false,        
        every head row/.style={before row=\toprule, after row=\midrule},
        every last row/.style={after row=\bottomrule},
        every col no 0/.style={fixed, precision=0},            
        every col no 1/.style={fixed, fixed zerofill, precision=2}, 
        every col no 2/.style={fixed, fixed zerofill, precision=0}, 
        every col no 3/.style={fixed, fixed zerofill, precision=2}, 
    }
}

\newcommand{\err}[1]{\textcolor{red}{#1}}
\crefname{table}{tab.}{tab.}

\title{Misura della caratteristica I-V di diodi a giunzione p-n al germanio e al silicio}
\author{Enrico Barbuio \\ 0001117553 \and Giacomo Cicala \\ 0001122965} 
\date{Turno 3 del 20 Novembre 2025, Tavolo 11}

\begin{document}
\maketitle

\renewcommand{\abstractname}{Abstract}
\begin{abstract}

	L'esperimento ha avuto come obiettivo la realizzazione di un circuito per la
	misura delle curve caratteristiche I-V di due diodi a giunzione p-n, uno al
	silicio e uno al germanio. I valori misurati della corrente di saturazione
	inversa e del prodotto del fattore di idealità con la tensione termica, per il
	diodo al silicio sono:

	\begin{equation*}
		{(I_{0})}_{Si} = ( 1 \pm 2 ) \ \unit{\nA} \hspace{2cm} {(\eta V_{T})}_{Si} = (47 \pm 7) \ \unit{\mV}
	\end{equation*}

	\noindent
	mentre per il diodo al germanio sono:

	\begin{equation*}
		(I_{0})_{Ge} = ( 1 \pm 4 ) \ \unit{\uA} \hspace{2cm} (\eta V_{T})_{Ge} = (4 \pm 2) \times 10^1 \ \unit{\mV}
	\end{equation*}
\end{abstract}

\section*{Introduzione}

Nell'esperimento svolto in laboratorio abbiamo realizzato un circuito
per la misura della caratteristica I-V di due diodi a giunzione p-n a
polarizzazione diretta, uno al silicio e uno al germanio. La caratteristica I-V
di un diodo non ideale è descritta dall'equazione di Shockley:

\begin{equation}
	I = I_{0} \left( e^{\frac{V}{\eta V_{T}}} - 1 \right)
	\label{eq:shockley}
\end{equation}

\noindent
dove $I$ è la corrente che attraversa il diodo, $V$ è la tensione ai suoi capi,
$I_{0}$ è la corrente di saturazione inversa e $V_{T}$ è la tensione termica.
Il termine $\eta$ è il fattore di idealità: questo tiene conto delle
deviazioni del diodo reale rispetto al modello teorico ideale di Shockley,
dovute principalmente a fenomeni di generazione termica e ricombinazione
all'interno della depletion region.

\section*{Apparato sperimentale e svolgimento}

Il circuito (\cref{fig:schema_elettrico}) è stato realizzato su una piastra
forata utilizzando un alimentatore a bassa tensione \texttt{TTi EB2025T}
(\cref{fig:alimentatore}) con uscita fissa a \qty{+5}{\V}, un potenziometro da
\qty{1}{\kohm} e due diodi a giunzione p-n, uno al silicio e uno al germanio. La tensione ai capi del diodo e la
corrente che lo attraversa sono state misurate rispettivamente con un
oscilloscopio da banco \texttt{GW Instek GOS-652G} (\cref{fig:oscilloscopio}) e
un multimetro \texttt{Fluke 175 True RMS Multimeter} (\cref{fig:multimetro}) in
modalità amperometro.

\begin{figure}[h!]
	\centering
	\begin{circuitikz}
		% Generatore
		\draw (0,0) coordinate (GND) to[voltage source, invert, l=$+5V$] ++(0,3) -- ++(1.5,0) coordinate (pot_up);

		% Potenziometro
		\draw (pot_up) to[potentiometer, name=P, l_=$R$] ++(0,-3) coordinate (pot_down);

		\draw (pot_down) to[short, -*] (GND) node[ground]{};

		\draw (P.wiper) -- ++(0.5,0) coordinate (tmp) -- (tmp |- 0,3) --
		++(1.5,0) coordinate(oscope_up) to[oscope, v=$V_{osc}$] (oscope_up |- GND) coordinate(oscope_down) to [short, -*] (pot_down);

		\draw (oscope_up) to[short, *-] ++(2,0) coordinate (vmeter_up) to[rmeterwa, t=V, v=$V_{mult}$] (vmeter_up |- GND) to [short, -*] (oscope_down);
	\end{circuitikz}
	\hfill
	\begin{circuitikz}
		% Generatore
		\draw (0,0) coordinate(GND) to[voltage source, invert, l=$+5V$] ++(0,3) -- ++(1.5,0) coordinate (pot_up);

		% Potenziometro
		\draw (pot_up) to[potentiometer, name=P, l_=$R$] ++(0,-3) coordinate (pot_down);

		\draw (pot_down) to[short, -*] (GND) node[ground]{};

		% Diodo
		\draw (P.wiper) -- ++(0.5,0) coordinate (tmp) -- (tmp |- 0,3) to[rmeterwa, t=A, i=$i$] ++(2,0) coordinate(diode_up) to[diode] (diode_up |- 0,0) coordinate (diode_down);

		% Tensione
		\draw (diode_up) to[short, *-] ++(2,0) coordinate (tmp) to[oscope, v=$V$] (tmp |- diode_down) to[short, -*] (diode_down);
		% Corrente
		\draw (diode_down) to [short, -*] (pot_down);
	\end{circuitikz}
	\caption{Schema elettrico del circuito realizzato.
		A sinistra la configurazione per la calibrazione dell'oscilloscopio;
		a destra quella per la misura della caratteristica I-V del diodo.}
	\label{fig:schema_elettrico}
\end{figure}

\begin{figure}[h!]
	\centering
	\begin{subfigure}[b]{0.32\textwidth}
		\centering
		\includegraphics[width=\linewidth]{images/alimentatore.png}
		\caption{Alimentatore}
		\label{fig:alimentatore}
	\end{subfigure}
	\hfill
	\begin{subfigure}[b]{0.32\textwidth}
		\centering
		\includegraphics[width=\linewidth]{images/oscilloscopio3.png}
		\caption{Oscilloscopio}
		\label{fig:oscilloscopio}
	\end{subfigure}
	\hfill
	\begin{subfigure}[b]{0.32\textwidth}
		\centering
		\includegraphics[width=\linewidth]{images/multimetro.png}
		\caption{Multimetro}
		\label{fig:multimetro}
	\end{subfigure}
	\caption{Strumentazione usata nell'apparato sperimentale:
		(a) Alimentatore  \texttt{TTi EB2025T}; (b) Oscilloscopio
		\texttt{GW Instek GOS-652G}; (c) Multimetro
		\texttt{Fluke 175 True RMS Multimeter}.}
	\label{fig:apparato_sperimentale}
\end{figure}

Prima di verificare l'equazione di Shockley è stato calibrato l'oscilloscopio
usando il multimetro come riferimento. Per fare ciò è stato realizzato il primo
circuito di \cref{fig:schema_elettrico}, in cui oscilloscopio e multimetro sono
in parallelo e misurano la stessa tensione. Variando la resistenza del
potenziometro sono stati acquisiti dieci punti con tensioni comprese tra
\qty{0.05}{\V} e \qty{0.8}{\V} (\cref{tab:calibrazione_strumenti}).

Per ricavare le caratteristiche I-V dei diodi si è realizzato il secondo
circuito di \cref{fig:schema_elettrico}, prima con il diodo al silicio e poi
con quello al germanio. In questa configurazione l'oscilloscopio è collegato in
parallelo al diodo, misurando la differenza di potenziale tra i due capi; il
multimetro invece è collegato in serie, in modo da misurare la corrente che lo
attraversa. Dopodiché, variando la resistenza del potenziometro e quindi la
tensione sul diodo, sono state acquisite diverse coppie di valori di tensione e
corrente per ciascun diodo. I dati acquisiti sono mostrati in
\cref{tab:data_silicio,tab:data_germanio} e \cref{fig:caratteristiche}.

\section*{Risultati e discussione}
Si riportano in \cref{tab:calibrazione_strumenti} i valori di tensione misurati con oscilloscopio e multimetro per effettuare la calibrazione degli strumenti.
\sisetup{
	table-number-alignment = center,
	table-figures-integer = 4,
	table-figures-decimal = 8
}

Le incertezze del multimetro sono state omesse in quanto trascurabili rispetto
a quelle dell'oscilloscopio. La risoluzione dell'oscilloscopio (e quindi
l'errore sulla lettura) $\sigma_{l}$ è stata calcolata come:
\begin{equation}
	\sigma_{l} = \frac{\text{F.S.}}{5}*(\text{\#tacchette apprezzabili})
\end{equation}
Come numero di tacchette apprezzabili si è considerato $1/2$.
Infine, l'errore totale associato ad ogni misura di tensione con l'oscilloscopio è stato calcolato come:

\begin{equation}
	\sigma_{osc} = \sqrt{\sigma_{l}^{2} + \sigma_{c}^{2}}
\end{equation}

\noindent
dove $\sigma_{c}$ è l'errore dichiarato dal costruttore e vale:

\begin{equation}
	\frac{\sigma_{c}}{V_{mis}} = 3\%
\end{equation}

\noindent
con $V_{mis}$ tensione misurata; l'errore sullo zero dell'oscilloscopio è stato
trascurato in quanto si è verificato che fosse molto più piccolo rispetto agli
altri errori, grazie ad un opportuno fondo scala di \qty{5}{\mV \per Div}.

\begin{figure}[h!]
	\centering
	\includegraphics[width=0.5\textwidth]{calibrazione.pdf}
	\caption{Fit lineare dei dati di calibrazione tra multimetro e
		oscilloscopio. Sulle ascisse sono riportati i valori di tensione misurati
		con il multimetro, sulle ordinate quelli misurati con l'oscilloscopio.}
	\label{fig:fit_calibrazione}
\end{figure}

\`E stato eseguito un fit lineare sui dati di calibrazione
(\cref{fig:fit_calibrazione}) con la seguente relazione:
\begin{equation}
	V_{osc} = a + b \cdot V_{mult}
\end{equation}
ottenendo come parametri:
\begin{equation*}
	a = (-6 \pm 3) \text{ mV} \hspace{2cm} b = 1.014 \pm 0.017
\end{equation*}
Questi indicano che il coefficiente angolare $b$ è compatibile con l'unità,
mentre l'intercetta non è compatibile con lo zero. Le successive misure di
tensione fatte con l'oscilloscopio sono state corrette per tenere conto di
questo offset:
\begin{equation*}
	V = V_{mis} - a \hspace{2cm} \sigma_V = \sqrt{{\sigma_l}^2 + {\sigma_c}^2 + {\sigma_a}^2}
\end{equation*}

In \cref{fig:caratteristiche} si possono vedere i punti misurati con la
rispettiva funzione caratteristica. Il fit del silicio è stato svolto su tutti
i dati; mentre per il germanio ci si è ristretti agli 8 punti nel range
\qty{0}{\mV} -- \qty{180}{\mV}. Questo perchè oltre tale valore la
caratteristica si discosta dall'andamento esponenziale previsto
dall'equazione di Shockley a causa dell'effetto di resistenza in serie del diodo, che mentre per quello al silicio è trascurabile, per quello al germanio diventa significativo per effetto della corrente più elevata già a basse tensioni.

\begin{figure}[h]
	\includegraphics[width=\textwidth]{caratteristiche.pdf}
	\caption{Caratteristica I-V misurata per diodo al silicio (a sinistra) e al
		germanio (a destra). Asse delle ordinate in scala logaritmica. Il fit è
		ristretto alla regione in cui sono graficate le funzioni.}
	\label{fig:caratteristiche}
\end{figure}

Dall'analisi risulta:
\begin{equation*}
	{(I_{0})}_{Si} = ( 1 \pm 2 ) \ \unit{\nA} \hspace{2cm} {(\eta V_{T})}_{Si} = (47 \pm 7) \ \unit{\mV}
\end{equation*}

\noindent
per il silicio; mentre per il diodo al germanio:

\begin{equation*}
	(I_{0})_{Ge} = ( 1 \pm 4 ) \ \unit{\uA} \hspace{2cm} (\eta V_{T})_{Ge} = (4 \pm 2) \times 10^1 \ \unit{\mV}
\end{equation*}

\section*{Conclusioni}
Si può notare come all'interno degli intervalli di fit, le caratteristiche siano compatibili con  l'andamento esponenziale previsto
dall'equazione di Shockley.
Infine, i valori misurati della corrente di saturazione
	inversa e del prodotto del fattore di idealità con la tensione termica, per il
	diodo al silicio sono:

	\begin{equation*}
		{(I_{0})}_{Si} = ( 1 \pm 2 ) \ \unit{\nA} \hspace{2cm} {(\eta V_{T})}_{Si} = (47 \pm 7) \ \unit{\mV}
	\end{equation*}

	\noindent
	mentre per il diodo al germanio sono:

	\begin{equation*}
		(I_{0})_{Ge} = ( 1 \pm 4 ) \ \unit{\uA} \hspace{2cm} (\eta V_{T})_{Ge} = (4 \pm 2) \times 10^1 \ \unit{\mV}
	\end{equation*}



\appendix
\section{Misure tabulate}

\begin{table}[h]
	\centering
	\begin{tabular}{
			S[table-format=1.4] % 4 decimali per il multimetro (es. 0.0568)
			S[table-format=1.3] % 3 decimali max per l'oscilloscopio
			S[table-format=1.2] % Scala
			S[table-format=1.3] % Risoluzione
			S[table-format=1.3] % Errore (varia tra 2 e 3 decimali)
		}
		\toprule
		{ \boldmath $V_{mult}$ }   &
		{ \boldmath $V_{osc}$ }    &
		{ \boldmath $F.S._{osc}$ } &
		{ \boldmath $\sigma_{l}$ } &
		{ \boldmath $\sigma_{osc}$ }                                           \\

		{ (V) }                    & { (V) } & { (V/div) } & { (V) } & { (V) } \\
		\midrule
		0.0568                     & 0.052   & 0.02        & 0.002   & 0.003   \\
		0.1453                     & 0.140   & 0.05        & 0.005   & 0.007   \\
		0.2288                     & 0.220   & 0.05        & 0.005   & 0.008   \\
		0.2980                     & 0.300   & 0.10        & 0.010   & 0.013   \\
		0.3773                     & 0.380   & 0.10        & 0.010   & 0.015   \\
		0.4537                     & 0.460   & 0.10        & 0.010   & 0.017   \\
		0.5316                     & 0.540   & 0.10        & 0.010   & 0.019   \\
		0.6350                     & 0.64    & 0.20        & 0.020   & 0.03    \\
		0.7090                     & 0.72    & 0.20        & 0.020   & 0.03    \\
		0.8150                     & 0.80    & 0.20        & 0.020   & 0.03    \\
		\bottomrule
	\end{tabular}
	\caption{Coppie di punti di tensione misurate con multimetro e oscilloscopio. La prima colonna riporta la tensione misurata con il multimetro, la seconda quella misurata con l'oscilloscopio, la terza il fondo scala dell'oscilloscopio, la quarta la risoluzione dell'oscilloscopio e l'ultima l'errore totale associato alla misura con l'oscilloscopio.}
	\label{tab:calibrazione_strumenti}
\end{table}

\begin{table}[h]
	\centering
	\begin{minipage}[t]{0.48\textwidth}
		\centering
		\pgfplotstabletypeset[
			physicsTable,
			display columns/0/.style={column name=\textbf{\boldmath $V$} (\unit{\mV})},
			display columns/1/.style={column name=\textbf{\boldmath $I$} (\unit{\mA})},
			display columns/2/.style={column name=\textbf{\boldmath $\sigma_V$} (\unit{\mV})},
			display columns/3/.style={column name=\textbf{\boldmath $\sigma_I$} (\unit{\mA})},
		]{dati_silicio.txt}
		\caption{Dati del diodo al silicio.}
		\label{tab:data_silicio}
	\end{minipage}
	\hfill
	\begin{minipage}[t]{0.48\textwidth}
		\centering
		\pgfplotstabletypeset[
			physicsTable,
			display columns/0/.style={column name=\textbf{\boldmath $V$} (\unit{\mV})},
			display columns/1/.style={column name=\textbf{\boldmath $I$} (\unit{\mA})},
			display columns/2/.style={column name=\textbf{\boldmath $\sigma_V$} (\unit{\mV})},
			display columns/3/.style={column name=\textbf{\boldmath $\sigma_I$} (\unit{\mA})},
		]{dati_germanio.txt}
		\caption{Dati del diodo al germanio.}
		\label{tab:data_germanio}
	\end{minipage}
\end{table}

\end{document}
